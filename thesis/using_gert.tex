\chapter{How to Use GERT}

It would be silly for this thesis to present a software toolkit and then
neglect to show how to use it.

\section{API Inspiration}
GERT's API is based on Arduino's because of its remarkable simplicity. In order to
get started with GERT on a Freescale iMX6 SOC, the programmer must only implement three functions: \textit{user\_init},
\textit{user\_loop} and \textit{irq}. \textit{user\_init} should contain code that is run only once
on startup, \textit{user\_loop} should contain the main event loop of the embedded program, and
\textit{irq} is the IRQ handler. The programmer must make sure that \textit{irq} never executes
a blocking Go function or allocates memory on the Go heap with \textit{make()}
since interrupts are allowed to be serviced while the world is stopped for
a garbage collection. This effectively limits interrupt service routines to toggling
global boolean flags and deterministic memory reads/writes.

\section{Building a GERT Program}

\begin{figure}[h]
\begin{center}
  \includegraphics[scale=0.5]{dirtree}
\end{center}
  \caption{GERT Program Directory Layout} \label{fig:dtree}
\end{figure}

GERT is a modified version of the Go runtime but it needs a bootloader
and entry point to start on bare metal. The directory structure of a GERT
program is shown in fig. \ref{fig:dtree}. The boot directory contains the GERT
bootloader as well as linker script for it. Userprog.go and irq.go contain
the user-implemented functions as well as the user-implemented irq handler.
Kernel.go does three mundane things: it defines a new entry point for the Go runtime
(which just sets a flag so the runtime knows it is booting on bare metal),
it also finishes booting additional CPU's, and it configures the ARM Generic Interrupt Controller for normal operation. The makefile
is responsible for stitching the bootloader and GERT program together into an
sdcard image for u-boot. It uses \textit{go build} to build the GERT program
with the modified Go runtime and then it inserts the GERT program as a binary blob
into the boot loader's data section. The makefile also includes a target which writes
u-boot and the final GERT binary to an sdcard.

\section{Design Considerations}
Every SOC has a different memory map and peripherals, so GERT must be adjusted
accordingly. In order to change the link address of GERT, pass "-T <link address>"
as a link flag into \textit{go build}. The link address of the bootloader also needs
to be changed too inside link.ld. It is good practice to link the bootloader in an
area of RAM that can be reclaimed once GERT enables paging.

\section{Writing Drivers}
GERT imposes no driver model on the programmer. All drivers in
GERT should be written as normal Go code in the best style for
the intended application. GERT exposes no safe methods for reading
or writing device memory so any MMIO peripherals in the SOC must be
carefully programmed using the \textit{unsafe} package. Most MMIO
peripherals arrange their registers contiguously in memory so they
can be represented with a Go struct, which requires only one unsafe cast
for initial assignment.

GERT currently comes with an example driver
library in the form of a package called \textit{embedded}. The embedded package is not intrinsic to GERT's
functionality, nor was it optimized for performance in any way. The embedded package
just aims to provide a template for how drivers can be written in the Go language.
It only functions for the Freescale i.MX6 and includes drivers for the UART, SPI, PWM, GIC, USDHC, GPIO, and GPT peripherals.
The embedded package also includes a generic implementation of the FAT32 file system, which is
layered on top of a \textit{read} and \textit{write} function that the programmer can define.


